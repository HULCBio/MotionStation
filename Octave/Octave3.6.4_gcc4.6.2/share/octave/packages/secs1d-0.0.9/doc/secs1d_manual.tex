\documentclass[9pt]{amsart} 
\usepackage{geometry} 
\geometry{a4paper}
\usepackage{graphicx} 
\usepackage{amssymb} 
\usepackage{epstopdf} 
\usepackage{cprotect} 
\usepackage{float} 
\floatstyle{plain} 
\newfloat{demo}{thp}{dem} 
\floatname{demo}{Demo} 
\newfloat{demoout}{thp}{deo} 
\floatname{demoout}{Demo Output} 
\newcommand{\unit}[1]{\mathrm{#1}}
\newcommand{\electronvolt}{\unit{eV}}
\newcommand{\kelvin}{\unit{K}}
\newcommand{\nano}{\unit{n}}
\newcommand{\meter}{\unit{m}}
\newcommand{\second}{\unit{s}}
\newcommand{\volt}{\unit{V}}
\newcommand{\Ampere}{\unit{A}}


\title{secs1d}
\author{Carlo de Falco \and Riccardo Sacco}
\begin{document}
\maketitle
\titlepage
\tableofcontents

\begin{table}
\caption{secs1d Package Description}
\centering
\begin{tabular}{|l|l|}
\hline
{\bf Name: } & secs1d\\  \hline
{\bf Description: } &
A Drift-Diffusion simulator for 1d semiconductor devices\\  \hline
{\bf Version: } & 0.0.9\\  \hline
{\bf Release Date: } & 2012-03-25\\  \hline
{\bf Author: } & Carlo de Falco\\   \hline
{\bf Maintainer: } & Carlo de Falco\\  \hline
{\bf License: } & GPL version 2 or later\\  \hline
{\bf Depends on: } &
octave ($>=$ 3.0.0), bim ($>=$ 0.0.0), \\  \hline
{\bf Autoload: } &No\\  \hline
\end{tabular}
\end{table}
\clearpage

\part{Mathematical models}

\section{Full model}
\subsection{Conservation laws}

\begin{equation}\label{eq:conservation}
\left\{
\begin{array}{ll}
-\lambda^{2}\mathrm{div}\ \left(\varepsilon_{r} \mathrm{grad}\ 
\varphi \right) = p - n + N_{D} - N_{A} \\[5mm]
-\mathrm{div}\ \left(J_{n} \right) + R_{n} \, n = G_{n} \\[5mm]
\phantom{-}\mathrm{div}\ \left(J_{p} \right) + R_{p} \, p = G_{p}
\end{array}
\right.
\end{equation}

\section{Constitutive relations}

\subsection{Currents}

\begin{equation}\label{eq:currents}
\left\{
\begin{array}{ll}
J_{n} = \phantom{-}\mu_{n} \left( \mathrm{grad}\ n - n\ \mathrm{grad}\ \varphi\right) 
\\[5mm]
J_{p} = -\mu_{p} \left( \mathrm{grad}\ p + p\ \mathrm{grad}\ \varphi\right)  
\end{array}
\right.
\end{equation}

\subsection{Mobilities}

\begin{equation}\label{eq:mobilities}
\left\{
\begin{array}{ll}
\mu_{n} = \displaystyle \frac{2\bar{\mu}_{n}}
{1 + \sqrt{1 + 4 \left( \displaystyle \frac{\bar{\mu}_{n}|E|}{v_{sat,n}}\right)^{2}}}
; \qquad
\bar{\mu}_{n} = \mu_{min, n} + 
\displaystyle \frac{\mu_{0,n} - \mu_{min,n}}
{1 +\displaystyle \left(\frac{N_{D}+N_{A}}{N_{ref,n}}\right)^{\beta_{n}}}
\\[10mm]
\mu_{p} = \displaystyle \frac{2\bar{\mu}_{p}}
{1 + \sqrt{1 + 4 \left( \displaystyle \frac{\bar{\mu}_{p}|E|}{v_{sat,p}}\right)^{2}}}
; \qquad
\bar{\mu}_{p} = \mu_{min, p} + 
\displaystyle \frac{\mu_{0,p} - \mu_{min,p}}
{1 +\displaystyle \left(\frac{N_{D}+N_{A}}{N_{ref,p}}\right)^{\beta_{p}}}
\end{array}
\right.
\end{equation}

\subsection{Production terms}

\begin{equation}\label{eq:recombination}
\left\{
\begin{array}{ll}
R_{n} = \displaystyle 
\frac{p}{\tau_{n} (p + \theta) + \tau_{p} (n + \theta)}
+ p \left(C_{n} n + C_{p} p \right)
\\[5mm]
R_{p} = \displaystyle 
\frac{n}{\tau_{n} (p + \theta) + \tau_{p} (n + \theta)}
+ n \left (C_{n} n + C_{p} p \right)
\end{array}
\right.
\end{equation}

\begin{equation}\label{eq:generation}
G_{n} = G_{p} = 
\displaystyle 
\frac{\theta^{2}}{\tau_{n} (p + \theta) + \tau_{p} (n + \theta)}
+ \theta^{2} \left(C_{n} n + C_{p} p \right)
+ \left(\alpha_{n} |J_{n}|+ \alpha_{p} |J_{p}| \right)
\end{equation}

%\subsection{Ionization coefficients}

\begin{equation}\label{eq:ioniz_coeff}
\left\{
\begin{array}{ll}
\alpha_{n} = \displaystyle 
\alpha_{n}^{\infty} \exp \left( -\frac{E_{crit,n}}{|E|} \right)
\\[5mm]
\alpha_{p} = \displaystyle 
\alpha_{p}^{\infty} \exp \left( -\frac{E_{crit,p}}{|E|} \right)
\end{array}
\right.
\end{equation}

\newpage

\section{Simplified model used for Newton's method}
\subsection{Conservation laws}

\begin{equation}\label{eq:conservationN}
\left\{
\begin{array}{ll}
-\lambda^{2}\mathrm{div}\ \left(\varepsilon_{r} 
\mathrm{grad}\ \varphi \right) = p - n + N_{D} - N_{A} \\[5mm]
-\mathrm{div}\ \left(J_{n} \right) + R_{n} \, n = G_{n} \\[5mm]
\phantom{-}\mathrm{div}\ \left(J_{p} \right) + R_{p} \, p = G_{p}
\end{array}
\right.
\end{equation}

\section{Constitutive relations}

\subsection{Currents}

\begin{equation}\label{eq:currentsN}
\left\{
\begin{array}{ll}
J_{n} = \phantom{-}\mu_{n} \left( \mathrm{grad}\ n - n\ \mathrm{grad}\ \varphi\right) 
\\[5mm]
J_{p} = -\mu_{p} \left( \mathrm{grad}\ p + p\ \mathrm{grad}\ \varphi\right)  
\end{array}
\right.
\end{equation}

\subsection{Production terms}

\begin{equation}\label{eq:recombinationN}
\left\{
\begin{array}{ll}
R_{n} = \displaystyle \frac{p}{\tau_{n} (p + \theta) + \tau_{p} (n + \theta)}
+ p \left(C_{n} n + C_{p} p \right)
\\[5mm]
R_{p} = \displaystyle \frac{n}{\tau_{n} (p + \theta) + \tau_{p} (n + \theta)}
+ n \left (C_{n} n + C_{p} p \right)
\end{array}
\right.
\end{equation}

\begin{equation}\label{eq:generationN}
G_{n} = G_{p} = 
\displaystyle \frac{\theta^{2}}{\tau_{n} (p + \theta) + \tau_{p} (n + \theta)}
+ \theta^{2} \left(C_{n} n + C_{p} p \right)
\end{equation}

\newpage

\section{Scaling factors/adimensional parameters}

Given any generic quantity $u$ having units $U$, we
define the {\em scaled} quantity $\widehat{u}$ as
$$
\widehat{u} : = \displaystyle \frac{u}{\overline{u}}
$$
where $\overline{u}$ is the scaling factor associated with $u$
and having the same units as $u$. 

\begin{table}[h!]
\begin{center}
\begin{tabular}{lll}\hline
\textbf{Scaling factor}	& \textbf{Value} & \textbf{Units}\\ \hline
$\overline{x}$            & $L$            & $\meter$ \\[1mm]
$\overline{n}$            & $\| N_D^+ - N_A^-\|_{L^{\infty}(0,L)}$  
& $\meter^{-3}$ \\[1mm]
$\overline{\varphi}$      & $K_B T / q \simeq 26 \cdot 10^{-3}$ 
& $\volt$ \\[1mm]
$\overline{\mu}$          & $\max\left\{ \mu_{0,n}, \, \mu_{0,p}\right\}$ 
&  $\meter^2\,\volt^{-1}\,\second^{-1}$ \\[1mm]
$\overline{t}$          & $\overline{x}^2/(\overline{\mu} \, \overline{\varphi})$
&  $\second$ \\[1mm]
$\overline{R}$          & $\overline{n}/\overline{t}$
&  $\meter^{-3} \second^{-1}$ \\[1mm]
$\overline{E}$          & $\overline{\varphi}/\overline{x}$
&  $\volt \meter^{-1}$  \\[1mm]
$\overline{J}$          & $q \, \overline{\mu} \, \overline{n} \, 
\overline{E}$ &  $\Ampere \meter^{-2}$  \\[1mm]
$\overline{\alpha}$ & $\overline{x}^{-1}$ & $\meter^{-1}$ \\[1mm]
$\overline{C}_{Au}$ & $\overline{R}/\overline{n}^3$ & 
$\meter^{6} \second^{-1}$ \\[1mm]
\hline
\end{tabular}
\caption{Scaling factors for the Drift-Diffusion model equations.}
\label{tab:model_param_1d}
\end{center}
\end{table}

We also introduce the following adimensional numbers
$$
\lambda^2:= \displaystyle \frac{\varepsilon_0 \overline{\varphi}}
{q \, \overline{n} \, \overline{x}^2}, \qquad
\theta:= \displaystyle \frac{n_i}{\overline{n}}
$$
having the meaning of squared normalized Debye length and
normalized intrinsic concentration, respectively.

\part{Function reference}

\section{Drift-Diffusion solvers}

\subsection{secs1d\_dd\_gummel\_map}
\begin{verbatim}


 [n, p, V, Fn, Fp, Jn, Jp, it, res] = secs1d_dd_gummel_map (x, D, Na, Nd, 
                                                       pin, nin, Vin, Fnin, 
                                                       Fpin, l2, er, u0n, 
                                                       uminn, vsatn, betan, 
                                                       Nrefn, u0p, uminp, vsatp, 
                                                       betap, Nrefp, theta, tn, tp, 
                                                       Cn, Cp, an, ap, Ecritnin, Ecritpin, 
                                                       toll, maxit, ptoll, pmaxit)         

 This function solves the scaled stationary bipolar DD 
 equation system using Gummel algorithm

     input: 
            x                        spatial grid
            D, Na, Nd                doping profile
            pin                      initial guess for hole concentration
            nin                      initial guess for electron concentration
            Vin                      initial guess for electrostatic potential
            Fnin                     initial guess for electron Fermi potential
            Fpin                     initial guess for hole Fermi potential
            l2                       scaled Debye length squared
            er                       relative electric permittivity
            u0n, uminn, vsatn, Nrefn electron mobility model coefficients
            u0p, uminp, vsatp, Nrefp hole mobility model coefficients
            theta                    intrinsic carrier density
            tn, tp, Cn, Cp, 
            an, ap, 
            Ecritnin, Ecritpin       generation recombination model parameters
            toll                     tolerance for Gummel iterarion convergence test
            maxit                    maximum number of Gummel iterarions
            ptoll                    convergence test tolerance for the non linear
                                     Poisson solver
            pmaxit                   maximum number of Newton iterarions

     output: 
             n     electron concentration
             p     hole concentration
             V     electrostatic potential
             Fn    electron Fermi potential
             Fp    hole Fermi potential
             Jn    electron current density
             Jp    hole current density
             it    number of Gummel iterations performed
             res   total potential increment at each step


\end{verbatim}




\subsection{Demo 1 for unction secs1d\_dd\_gummel\_map}
\begin{verbatim}

 % physical constants and parameters
 secs1d_physical_constants;
 secs1d_silicon_material_properties;
 
 % geometry
 L  = 10e-6;          % [m] 
 xm = L/2;
 
 Nelements = 1000;
 x         = linspace (0, L, Nelements+1)';
 sinodes   = [1:length(x)];
 
 % dielectric constant (silicon)
 er = esir * ones (Nelements, 1);
 
 % doping profile [m^{-3}]
 Na = 1e23 * (x <= xm);
 Nd = 1e23 * (x > xm);
 
 % avoid zero doping
 D  = Nd - Na;  
  
 % initial guess for n, p, V, phin, phip
 V_p = -1;
 V_n =  0;
 
 Fp = V_p * (x <= xm);
 Fn = Fp;
 
 p = abs (D) / 2 .* (1 + sqrt (1 + 4 * (ni./abs(D)) .^2)) .* (x <= xm) + ...
     ni^2 ./ (abs (D) / 2 .* (1 + sqrt (1 + 4 * (ni ./ abs (D)) .^2))) .* (x > xm);
 
 n = abs (D) / 2 .* (1 + sqrt (1 + 4 * (ni ./ abs (D)) .^ 2)) .* (x > xm) + ...
     ni ^ 2 ./ (abs (D) / 2 .* (1 + sqrt (1 + 4 * (ni ./ abs (D)) .^2))) .* (x <= xm);
 
 V = Fn + Vth * log (n / ni);
 
 % scaling factors
 xbar = L;                       % [m]
 nbar = norm(D, 'inf');          % [m^{-3}]
 Vbar = Vth;                     % [V]
 mubar = max (u0n, u0p);         % [m^2 V^{-1} s^{-1}]
 tbar = xbar^2 / (mubar * Vbar); % [s]
 Rbar = nbar / tbar;             % [m^{-3} s^{-1}]
 Ebar = Vbar / xbar;             % [V m^{-1}]
 Jbar = q * mubar * nbar * Ebar; % [A m^{-2}]
 CAubar = Rbar / nbar^3;         % [m^6 s^{-1}]
 abar = 1/xbar;                  % [m^{-1}]
 
 % scaling procedure
 l2 = e0 * Vbar / (q * nbar * xbar^2);
 theta = ni / nbar;
 
 xin = x / xbar;
 Din = D / nbar;
 Nain = Na / nbar;
 Ndin = Nd / nbar;
 pin = p / nbar;
 nin = n / nbar;
 Vin = V / Vbar;
 Fnin = Vin - log (nin);
 Fpin = Vin + log (pin);
 
 tnin = tn / tbar;
 tpin = tp / tbar;
 
 u0nin = u0n / mubar;
 uminnin = uminn / mubar;
 vsatnin = vsatn / (mubar * Ebar);
 
 u0pin = u0p / mubar;
 uminpin = uminp / mubar;
 vsatpin = vsatp / (mubar * Ebar);
 
 Nrefnin = Nrefn / nbar;
 Nrefpin = Nrefp / nbar;
 
 Cnin     = Cn / CAubar;
 Cpin     = Cp / CAubar;
 
 anin     = an / abar;
 apin     = ap / abar;
 Ecritnin = Ecritn / Ebar;
 Ecritpin = Ecritp / Ebar;
 
 % tolerances for convergence checks
 toll  = 1e-3;
 maxit = 1000;
 ptoll = 1e-12;
 pmaxit = 1000;
 
 % solve the problem using the full DD model
 [nout, pout, Vout, Fnout, Fpout, Jnout, Jpout, it, res] = ...
       secs1d_dd_gummel_map (xin, Din, Nain, Ndin, pin, nin, Vin, Fnin, Fpin, ...
                             l2, er, u0nin, uminnin, vsatnin, betan, Nrefnin, ...
 	                       u0pin, uminpin, vsatpin, betap, Nrefpin, theta, ...
 		               tnin, tpin, Cnin, Cpin, anin, apin, ...
 		               Ecritnin, Ecritpin, toll, maxit, ptoll, pmaxit); 
 
 % Descaling procedure
 n    = nout*nbar;
 p    = pout*nbar;
 V    = Vout*Vbar;
 Fn   = V - Vth*log(n/ni);
 Fp   = V + Vth*log(p/ni);
 dV   = diff(V);
 dx   = diff(x);
 E    = -dV./dx;
 
 % band structure
 Efn  = -Fn;
 Efp  = -Fp;
 Ec   = Vth*log(Nc./n)+Efn;
 Ev   = -Vth*log(Nv./p)+Efp;
 
 plot (x, Efn, x, Efp, x, Ec, x, Ev)
 legend ('Efn', 'Efp', 'Ec', 'Ev')
 axis tight
\end{verbatim}

\begin{figure}\centering
\includegraphics[width=.7\linewidth]{function/images/secs1d_dd_gummel_map_205.png}
\caption{Figure produced by demo number 1 for function secs1d\_dd\_gummel\_map}
\label{fig:secs1d_dd_gummel_map_figure_1}
\end{figure}
\clearpage


\subsection{secs1d\_dd\_newton}
\begin{verbatim}


 [n, p, V, Fn, Fp, Jn, Jp, it, res] = secs1d_dd_newton (x, D, Vin, nin, 
                                                        pin, l2, er, un, 
                                                        up, theta, tn, tp, 
                                                        Cn, Cp, toll, maxit)

 Solve the scaled stationary bipolar DD equation system using Newton's method

     input: 
       x                spatial grid
       D                doping profile
       pin              initial guess for hole concentration
       nin              initial guess for electron concentration
       Vin              initial guess for electrostatic potential
       l2               scaled Debye length squared
       er               relative electric permittivity
       un               electron mobility model coefficients
       up               electron mobility model coefficients
       theta            intrinsic carrier density
       tn, tp, Cn, Cp   generation recombination model parameters
       toll             tolerance for Gummel iterarion convergence test
       maxit            maximum number of Gummel iterarions

     output: 
       n     electron concentration
       p     hole concentration
       V     electrostatic potential
       Fn    electron Fermi potential
       Fp    hole Fermi potential
       Jn    electron current density
       Jp    hole current density
       it    number of Gummel iterations performed
       res   total potential increment at each step


\end{verbatim}




\subsection{Demo 1 for function secs1d\_dd\_newton}
\begin{verbatim}

 % physical constants and parameters
 secs1d_physical_constants;
 secs1d_silicon_material_properties;
 
 % geometry
 L  = 1e-6; % [m] 
 x  = linspace (0, L, 10)';
 sinodes = [1:length(x)];
 
 % dielectric constant (silicon)
 er = esir * ones (numel (x) - 1, 1);
 
 % doping profile [m^{-3}]
 Na = 1e20 * ones(size(x));
 Nd = 1e24 * ones(size(x));
 D  = Nd-Na;  
 
 % externally applied voltages
 V_p = 10;
 V_n = 0;
  
 % initial guess for phin, phip, n, p, V
 Fp = V_p * (x <= L/2);
 Fn = Fp;
 
 p = abs(D)/2.*(1+sqrt(1+4*(ni./abs(D)).^2)).*(D<0)+...
     ni^2./(abs(D)/2.*(1+sqrt(1+4*(ni./abs(D)).^2))).*(D>0);
 
 n = abs(D)/2.*(1+sqrt(1+4*(ni./abs(D)).^2)).*(D>0)+...
     ni^2./(abs(D)/2.*(1+sqrt(1+4*(ni./abs(D)).^2))).*(D<0);
 
 V  = Fn + Vth*log(n/ni);

 % scaling factors
 xbar = L;                         % [m]
 nbar = norm(D, 'inf');            % [m^{-3}]
 Vbar = Vth;                       % [V]
 mubar = max(u0n, u0p);            % [m^2 V^{-1} s^{-1}]
 tbar = xbar^2/(mubar*Vbar);       % [s]
 Rbar = nbar/tbar;                 % [m^{-3} s^{-1}]
 Ebar = Vbar/xbar;                 % [V m^{-1}]
 Jbar = q*mubar*nbar*Ebar;         % [A m^{-2}]
 CAubar = Rbar/nbar^3;             % [m^6 s^{-1}]
 abar = xbar^(-1);                 % [m^{-1}]
 
 % scaling procedure
 l2 = e0*Vbar/(q*nbar*xbar^2);     
 theta = ni/nbar;                  
 
 xin = x/xbar;
 Din = D/nbar;
 Nain = Na/nbar;
 Ndin = Nd/nbar;
 pin = p/nbar;
 nin = n/nbar;
 Vin = V/Vbar;
 Fnin = Vin - log(nin);
 Fpin = Vin + log(pin);
 
 tnin = tn/tbar;
 tpin = tp/tbar;
 
 % mobility model accounting scattering from ionized impurities
 u0nin = u0n/mubar;
 uminnin = uminn/mubar;
 vsatnin = vsatn/(mubar*Ebar);
 
 u0pin = u0p/mubar;
 uminpin = uminp/mubar;
 vsatpin = vsatp/(mubar*Ebar);
 
 Nrefnin = Nrefn/nbar;
 Nrefpin = Nrefp/nbar;
 
 Cnin     = Cn/CAubar;
 Cpin     = Cp/CAubar;
 
 anin     = an/abar;
 apin     = ap/abar;
 Ecritnin = Ecritn/Ebar;
 Ecritpin = Ecritp/Ebar;
 
 % tolerances for convergence checks
 ptoll = 1e-12;
 pmaxit = 1000;
 
 % solve the problem using the Newton fully coupled iterative algorithm
 [nout, pout, Vout, Fnout, Fpout, Jnout, Jpout, it, res] = secs1d_dd_newton (xin, Din, 
                                                                Vin, nin, pin, l2, er, 
                                                                u0nin, u0pin, theta, tnin, 
                                                                tpin, Cnin, Cpin, ptoll, pmaxit);
 % Descaling procedure
 n    = nout*nbar;
 p    = pout*nbar;
 V    = Vout*Vbar;
 Fn   = V - Vth*log(n/ni);
 Fp   = V + Vth*log(p/ni);
 dV   = diff(V);
 dx   = diff(x);
 E    = -dV./dx;
 
 % compute current densities 
 [Bp, Bm] = bimu_bernoulli (dV/Vth);
 Jn       =  q*u0n*Vth .* (n(2:end) .* Bp - n(1:end-1) .* Bm) ./ dx; 
 Jp       = -q*u0p*Vth .* (p(2:end) .* Bm - p(1:end-1) .* Bp) ./ dx;
 Jtot     =  Jn+Jp;
 
 % band structure
 Efn  = -Fn;
 Efp  = -Fp;
 Ec   = Vth*log(Nc./n)+Efn;
 Ev   = -Vth*log(Nv./p)+Efp;

 plot (x, Efn, x, Efp, x, Ec, x, Ev)
 legend ('Efn', 'Efp', 'Ec', 'Ev')
 axis tight
\end{verbatim}

\begin{figure}\centering
\includegraphics[width=.7\linewidth]{function/images/secs1d_dd_newton_819.png}
\caption{Figure produced by demo number 1 for function secs1d\_dd\_newton}
\label{fig:secs1d_dd_newton_figure_1}
\end{figure}
\clearpage


\section{Non-linear Poisson solver}
\subsection{secs1d\_nlpoisson\_newton}
\begin{verbatim}


 [V, n, p, res, niter] = secs1d_nlpoisson_newton (x, sinodes, Vin, nin, pin,
                                                  Fnin, Fpin, D, l2, er, toll, maxit)

     input:  
             x       spatial grid
             sinodes index of the nodes of the grid which are in the semiconductor subdomain
                     (remaining nodes are assumed to be in the oxide subdomain)
             Vin     initial guess for the electrostatic potential
             nin     initial guess for electron concentration
             pin     initial guess for hole concentration
             Fnin    initial guess for electron Fermi potential
             Fpin    initial guess for hole Fermi potential
             D       doping profile
             l2      scaled Debye length squared
             er      relative electric permittivity
             toll    tolerance for convergence test
             maxit   maximum number of Newton iterations

     output: 
             V       electrostatic potential
             n       electron concentration
             p       hole concentration
             res     residual norm at each step
             niter   number of Newton iterations


\end{verbatim}




\subsection{Demo 1 for function secs1d\_nlpoisson\_newton}
\begin{verbatim}

 secs1d_physical_constants
 secs1d_silicon_material_properties
 
 tbulk= 1.5e-6;
 tox = 90e-9;
 L = tbulk + tox;
 cox = esio2/tox;
 
 Nx  = 50;
 Nel = Nx - 1;
 
 x = linspace (0, L, Nx)';
 sinodes = find (x <= tbulk);
 xsi = x(sinodes);
 
 Nsi = length (sinodes);
 Nox = Nx - Nsi;
 
 NelSi   = Nsi - 1;
 NelSiO2 = Nox - 1;
 
 Na = 1e22;
 D = - Na * ones (size (xsi));
 p = Na * ones (size (xsi));
 n = (ni^2) ./ p;
 Fn = Fp = zeros (size (xsi));
 Vg = -10;
 Nv = 80;
 for ii = 1:Nv
     Vg = Vg + 0.2;
     vvect(ii) = Vg; 
     
     V = - Phims + Vg * ones (size (x));
     V(sinodes) = Fn + Vth * log (n/ni);
     
     % Scaling
     xs  = L;
     ns  = norm (D, inf);
     Din = D / ns;
     Vs  = Vth;
     xin   = x / xs;
     nin   = n / ns;
     pin   = p / ns;
     Vin   = V / Vs;
     Fnin  = (Fn - Vs * log (ni / ns)) / Vs;
     Fpin  = (Fp + Vs * log (ni / ns)) / Vs;
     
     er    = esio2r * ones(Nel, 1);
     l2(1:NelSi) = esi;
     l2    = (Vs*e0)/(q*ns*xs^2);
     
     % Solution of Nonlinear Poisson equation
     
     % Algorithm parameters
     toll  = 1e-10;
     maxit = 1000;
     
     [V, nout, pout, res, niter] = secs1d_nlpoisson_newton (xin, sinodes, 
                                                            Vin, nin, pin,
                                                            Fnin, Fpin, Din, l2,
                                                            er, toll, maxit);
 
     % Descaling
     n     = nout*ns;
     p     = pout*ns;
     V     = V*Vs;
     
     qtot(ii) = q * trapz (xsi, p + D - n);
 end
 
 vvectm = (vvect(2:end)+vvect(1:end-1))/2;
 C = - diff (qtot) ./ diff (vvect);
 plot(vvectm, C)
 xlabel('Vg [V]')
 ylabel('C [Farad]')
 title('C-V curve')
\end{verbatim}

\begin{figure}\centering
\includegraphics[width=.7\linewidth]{function/images/secs1d_nlpoisson_newton_85.png}
\caption{Figure produced by demo number 1 for function secs1d\_nlpoisson\_newton}
\label{fig:secs1d_nlpoisson_newton_figure_1}
\end{figure}
\clearpage


\section{Physical constants and material properties}
\subsection{secs1d\_physical\_constants.m}
\begin{verbatim}


 some useful physical constants 

 Kb       = Boltzman constant
 q        = quantum of charge
 e0       = permittivity of free space
 hplanck  = Plank constant
 hbar     = Plank constant by 2 pi
 mn0      = free electron mass
 T0       = temperature
 Vth 	   = thermal voltage


\end{verbatim}

\clearpage


\subsection{secs1d\_silicon\_material\_properties.m}
\begin{verbatim}


 material properties for silicon and silicon dioxide

 esir       = relative electric permittivity of silicon
 esio2r     = relative electric permittivity of silicon dioxide
 esi 	      = electric permittivity of silicon
 esio2      = electric permittivity of silicon dioxide
 mn         = effective mass of electrons in silicon
 mh         = effective mass of holes in silicon
 
 u0n        = low field electron mobility
 u0p        = low field hole mobility
 uminn      = parameter for doping-dependent electron mobility
 betan      = idem
 Nrefn      = idem
 uminp      = parameter for doping-dependent hole mobility
 betap      = idem
 Nrefp      = idem
 vsatn      = electron saturation velocity
 vsatp      = hole saturation velocity
 tp         = electron lifetime
 tn         = hole lifetime
 Cn         = electron Auger coefficient
 Cp         = hole Auger coefficient
 an         = impact ionization rate for electrons
 ap         = impact ionization rate for holes
 Ecritn     = critical field for impact ionization of electrons
 Ecritp     = critical field for impact ionization of holes 
 Nc         = effective density of states in the conduction band
 Nv         = effective density of states in the valence band
 Egap       = bandgap in silicon
 EgapSio2   = bandgap in silicon dioxide
 
 ni         = intrinsic carrier density
 Phims      = metal to semiconductor potential barrier


\end{verbatim}

\clearpage


\appendix
\section{Licence}
\input{COPYING.tex}

 
\end{document}