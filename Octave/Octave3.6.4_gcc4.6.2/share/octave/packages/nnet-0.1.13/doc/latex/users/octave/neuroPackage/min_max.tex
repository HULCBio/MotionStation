\subsection{min\_max}
\textit{min\_max} get the minimal and maximal values of an training input matrix. So the dimension of this matrix must be an RxN matrix where R is the number of input neurons and N depends on the number of training sets.\\

\noindent \textbf{\textcolor{brown}{Syntax:}}\\

\noindent mMinMaxElements = min\_max(RxN);\\

\noindent \textbf{\textcolor{brown}{Description:}}\\

\noindent RxN: R x N matrix of min and max values for R input elements with N columns\\ 

\noindent \textbf{\textcolor{brown}{Example:}}\\

\begin{equation}
	\left[
		\begin{array}{cc}
     	1 &  11 \\
    	0  & 21
   \end{array} 
	 \right]            = min\_max\left[ 
	 															   \begin{array}{ccccc}
	 															   3 & 1 & 3 & 5 & 11 \\
	 															   12& 0 & 21& 8 & 6  \\
	 															   \end{array}
	 															   	 \right]
\end{equation}

